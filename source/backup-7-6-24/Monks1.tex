<section xml:id="no-label">
<title>Fermat's Method for Finding Maxima and Minima</title>


\label{PSP:ferm-meth-find}

\setcounter{tasknumb}{0}

\begin{refsection}

\centerline{Kenneth M. Monks}
\centerline{Department of Mathematics}
\centerline{Front Range Community College -- Boulder County Campus,}
\centerline{Longmont, CO 80501}
\centerline{\texttt{kenneth.monks@frontrange.edu}}

\bigskip

A central theme of most introductory calculus courses is that of
<em>optimization</em>.  Given a real-valued function <m>f(x)</m>, one wishes
to find its maxima and minima on some specified interval of real
numbers.  Typically the backbone of this method is a theorem called
<em>Fermat's Theorem</em> or <em>Fermat's Stationary Point Theorem</em>
which is stated and illustrated below.


\vspace{0.1in}

\hrulefill

\begin{center}  {  Fermat's Theorem} 

\end{center}

\vspace{0.1in}
{\it 

If a real-valued function <m>f(x)</m> is differentiable on an interval
<m>(a,b)</m> and <m>f(x)</m> has a maximum or minimum at <m>c\in(a,b),</m> then
<m>f'(c)=0</m>.}   

\begin{center}
\begin{asy}
	size(300,150,keepAspect=false);  
    import graph;

    real f(real x)
    {
        return (x-1.2)^3-0.4x^2+1.5;
    }
    
    
    draw(graph(f,0.03,2,n=400),linewidth(0.6bp));
    
    draw((0.2,-0.03)--(0.2,0.03));
    label("<m>a</m>",(0.2,-0.03),S,fontsize(12pt));
    dot((0.2,f(0.2)),linewidth(4pt));
    
    draw((0.752,-0.03)--(0.752,0.03));
    label("<m>c</m>",(0.752,-0.03),S,fontsize(12pt));
    dot((0.752,f(0.752)),linewidth(4pt));
    
    draw((1.75,-0.03)--(1.75,0.03));
    label("<m>b</m>",(1.75,-0.03),S,fontsize(11pt));
    dot((1.75,f(1.75)),linewidth(4pt));
    
    draw((0.2,1.184)--(1.75,1.184),dashed);
    draw((0.752,0.06)--(0.752,f(0.752)),dashed);
    
    yaxis("<m>y</m>",-0.1, 1.3,fontsize(12pt),above=true);
	xaxis("<m>x</m>",-0.1, 2,fontsize(12pt),above=true);
\end{asy}

\end{center}

\hrulefill

\vspace{0.1in}

Most modern calculus courses use this theorem as the rationale behind
locating the maximum and minimum values of a continuous function
<m>f(x)</m> on an interval <m>[a,b]</m>, whose existence is guaranteed by the
Extreme Value Theorem.  The standard algorithm is to make a list of
the following <m>x</m>-values:
\begin{itemize}
    \item the endpoints <m>x=a</m> and <m>x=b</m>,
    \item any points <m>x\in(a,b)</m> such that <m>f(x)</m> is not differentiable, 
    \item and any points <m>x\in(a,b)</m> such that <m>f'(x)=0</m> (often called
      <em>stationary points</em> or <em>critical points</em>). 
    \end{itemize} Then, one calculates <m>f(x)</m> for each <m>x</m>-value,
    which produces a list of <m>y</m>-values.  Among this list, the biggest
    value of <m>f(x)</m> is the absolute maximum and the smallest is the
    absolute minimum.

\begin{task}
  Briefly explain how Fermat's Theorem serves as the basis for the
  optimization algorithm described above.
\end{task}

For the rest of this project, the method above will be referred to as
the <em>modern method</em>, in contrast to <em>Fermat's method</em>, which
we will now explore!

<subsubsection xml:id="Adequality">
<title>Fermat's Method\ldots and Descartes' Doubts!</title>


 Fermat's Theorem is so-called because it is traceable back to the
 ideas of Pierre de Fermat\footnote{Born in Beaumont-de-Lomagne in the
   south of France, Pierre de Fermat spent most of his life in
   Toulouse and Orl\'{e}ans, where he was educated as, and then worked
   as, a lawyer/jurist.  He found respite from his demanding career by
   pursuing his true love: mathematics!  Fermat championed the idea of
   <em>pure</em> mathematics; he was rarely motivated by problems
   pertaining to the physical world but rather loved mathematics for
   its own inherent beauty and challenge.} (1601--1665).  Nonetheless,
 it is fascinating to consider how different his method looks from the
 modern method!\footnote{Part of this difference, of course, has to do
   with the passage of time and the evolution of how we are expected
   to write mathematics.  However, part of it is also due to Fermat's
   unique personal style; he had a reputation for coming up with
   results in secret and then sending the result out into the
   mathematical community with no indication of how one might have
   come upon that, almost as a puzzle for the world to solve!
   Mathematics historian Victor Katz writes ``In many cases it is not
   known what, if any, proofs Fermat constructed nor is there always a
   systematic account of certain parts of his work.  Fermat often
   tantalized his correspondents with hints of his new methods for
   solving certain problems.  He would sometimes provide outlines of
   these methods, but his promises to fill in gaps `{\sf when leisure
     permits}' frequently remained unfulfilled.'' [\cite[page
   433]{Katz}]} His original writing, displayed below, is found in his
 1636 treatise <em>Method for the Study of Maxima and Minima</em>
 [\cite{FermatMaxMin}].  It should be noted the original was in
 Latin. Fermat's work in this document was translated into English by
 Jason Ross, working from the French translation by Tannery and Henry
 [\cite{Tannery}] (who modified some of Fermat's original notation).

\begin{source}

  Let <m>a</m> be an arbitrarily chosen unknown of the question (whether it
  has one, two, or three dimensions, as follows from the statement).
  We will express the maximum or minimum quantity in terms of <m>a</m>, by
  means of terms of any degree.  We will then substitute <m>a+e</m> for the
  primitive unknown <m>a</m>, and express the maximum or minimum quantity
  in terms containing <m>a</m> and <m>e</m> to any degree.  We will
  <em>ad-equate</em>, to speak like Diophantus,\footnote{Diophantus
    (c.~200{\sc ce}--c.~284{\sc ce}) was a mathematician in the city
    of Alexandria who wrote in Greek.  His word
    <m>\pi\alpha\rho\iota\sigma\acute{o}\tau\eta\zeta</m> (parisotes),
    meaning <em>approximately equal</em>, was translated into Latin as
    <em>adaequo</em> by the French mathematician Claude Gaspard Bachet
    de M\'{e}ziriac (1581--1638).  Fermat read Bachet's version of
    Diophantus' work [\cite{KatzAndFriends}].} the two expressions of
  the maximum and minimum quantity, and we will remove from them the
  terms common to both sides.  Having done this, it will be found that
  on both sides, all the terms will involve <m>e</m> or a power of <m>e</m>.  We
  will divide all the terms by <m>e</m>, or by a higher power of <m>e</m>, such
  that on at least one of the sides, <m>e</m> will disappear entirely.  We
  will then eliminate all the terms where <m>e</m> (or one of its powers)
  still exists, and we will consider the others equal, or if nothing
  remains on one of the sides, we will equate the added terms with the
  subtracted terms, which comes to be the same.  Solving this last
  equation will give the value of <m>a</m>, which will lead to the maximum
  or the minimum, in the original expression.

\end{source}

\begin{task}
  Compare and contrast the modern method with Fermat's method.  Can
  you find three similarities between them?  Can you find three
  differences between them?
 \end{task}

 Fermat himself was very pleased with this method, as he later made
 the following claim.

 \begin{source}
   It is impossible to give a more general method.
 \end{source}

Before we begin to analyze the algorithm described above to see
exactly what is happening, read it a second time.  Are you filled with
a bit of doubt as to whether or not this method is valid?  Are you
filled with a bit of curiosity as to where on earth this method might
have come from?  If so, you are in the best of company.  Rene
Descartes\footnote{Rene Descartes was born near Tours, France.  He is
  perhaps most famous today for his philosophical works, specifically
  as the writer of the phrase {``\sf je pense, donc je suis''} (in
  English, ``I think, therefore I am'') from his <em>Discourse on
    the Method</em> [\cite{Descartes}].  However, he also left mathematics
  with incredibly important and lasting advances.  He showed the power
  of symbolic algebra with regards to solving difficult geometric
  problems: he marked points in the plane using distances <m>x</m> and <m>y</m>
  measured along lines, much as we do today [\cite{Grabiner}].}
(1596--1650) read Fermat's treatise in 1638 after it was passed on to
him by Mersenne.\footnote{Marin Mersenne (1588--1648) was the central
  communications clearinghouse of a group of mathematicians and
  physicists. He would receive, copy, record, and distribute materials
  as they worked.  Fermat and Mersenne began a correspondence in
  1636.}  Descartes' response to Mersenne was somewhat dismissive; as
quoted in [\cite[p. 177]{FermatBio}], it included the remark ``{\sf
  \ldots if \ldots he speaks of wanting to send you still more papers,
  I beg of you to ask him to think them out more carefully than those
  preceding.}''

In this project, we aim to determine if Descartes was right, that
Fermat's method was not so carefully thought out.  Or, on the other
hand, was it a perfectly well-thought out method, but Fermat simply
chose to withhold the details of how he arrived at this method?

<subsection xml:id="no-label">
<title>Examples of Fermat's Method</title>


As one should begin any mathematical investigation, we first work out
a few examples.  In this section, we work through three problems that
Fermat himself used to demonstrate his method, to see if our modern
method reproduces the same results.




</subsubsection>
<subsubsection xml:id="First">
<title>First Example</title>




\begin{source}
Let us take an example: 

\begin{quote}
  <em>Divide the line <m>AC</m> at <m>E</m>, such that <m>AE\times EC</m> be a
    maximum.</em>
\end{quote}

\begin{center}
\begin{asy}
    size(300,30,keepAspect=false);
    
   draw((0.2,-0.03)--(0.2,0.03));
    label("<m>A</m>",(0.2,-0.03),S,fontsize(12pt));

    draw((0.752,-0.03)--(0.752,0.03));
    label("<m>E</m>",(0.752,-0.03),S,fontsize(12pt));
    
    draw((1.75,-0.03)--(1.75,0.03));
    label("<m>C</m>",(1.75,-0.03),S,fontsize(11pt));

    draw((0.2,0)--(1.75,0));
   
\end{asy}
\end{center}

Let us take <m>AC=b;</m> let <m>a</m> be one of the segments, and let the other
be <m>b-a</m>, and the product whose maximum we have to find is: <m>ba-a^2.</m>
Now let <m>a+e</m> be the first segment of <m>b</m>, the second <m>b-a-e</m>, and the
product of the two segments will be: <m>ba-a^2+be-2ae-e^2.</m>

It must be <em>co-equal to the preceding:</em> <m>ba-a^2;</m>

Removing the common terms: <m>be\sim 2ae+e^2;</m>

Dividing all the terms: <m>b\sim 2a+e;</m>

Remove <m>e</m>: <m>b=2a</m>.

To solve the problem, therefore, the half of <m>b</m> must be taken.
\end{source}

\begin{task}\label{Tomato}
  First, we solve the same problem using the modern method.  Denote by
  <m>b</m> the fixed total length of <m>AC</m> (just as Fermat did).  Then
  denote by <m>x</m> the length of <m>AE</m>, which implies <m>b-x</m> is the length
  of <m>EC</m>.

\begin{enumerate}[(a)]
    \item With the above notation, what is the function <m>f(x)</m> that we
      are trying to maximize?  What interval of <m>x</m> values are we
      considering?  
    \item Apply the modern method to find the absolute maximum of this
      function <m>f(x)</m>.  Does it confirm the result Fermat presents? 
\end{enumerate}
    
\end{task}

In practice, we tend to calculate the derivative of a function using
all of the standard slick and convenient formulas with which we have
become familiar: power rule, product rule, quotient rule, and chain
rule.  However, sometimes the limit definition of the derivative lends
a bit more insight into a problem than those other formulas lend.
Here our ``problem'' is trying to make sense of Fermat's method!

Specifically, for the next task we apply the limit definition of the
derivative, written as
<me>f'(x)=\lim_{\Delta x \to 0} \frac{f(x+\Delta x)-f(x)}{\Delta x}.</me>

\begin{task}
\begin{enumerate}[(a)]\label{comparison}
\item Take your function <m>f(x)</m> from the previous task, and again find
  the zeros of the derivative.  However, this time, don't worry about
  taking that limit so early in the process.  Instead, just write down
  the equation <me>\frac{f(x+\Delta x)-f(x)}{\Delta x}=0.</me> Simplify it
  as much as possible, and then, right at the very end, take the limit
  as <m>\Delta x</m> goes to zero.
    
\item Explain why the manipulation you performed above is equivalent
  to starting with <me>f(x+\Delta x)=f(x),</me> simplifying, dividing both
  sides by <m>\Delta x</m>, and then setting all the remaining occurrences
  of <m>\Delta x</m> to zero.
    
\item Now revisit Fermat's method.  When you compare your work to
  Fermat's, can you find similar steps?  Which symbol in the modern
  method corresponds to Fermat's <m>a</m>?  Which symbol in the modern
  method corresponds to Fermat's <m>e</m>?
\end{enumerate}
\end{task}


</subsubsection>
<subsubsection xml:id="Second">
<title>Second Example</title>

The result of the previous example is a slight rephrasing of what is
today known as the <em>vertex formula</em>: the fact that a quadratic
polynomial in <m>x</m> will achieve its absolute maximum or minimum when
<m>x</m> is the negative of the linear coefficient divided by twice the
leading coefficient.  Fermat's method worked out perfectly reasonably
in this case.  But perhaps it was only because the example was so
clean!  Let us examine a more complicated application of Fermat's
method.  This example was a followup note that Fermat wrote to his
original treatise, titled <em>On the Same Method</em> [\cite[page
126]{Tannery}].

\begin{source}
  By the means of my method, I would like <em>divide a given line
    <m>AC</m> at a point <m>B</m>, such that <m>AB^2\times BC</m> be the maximum</em> of
  all solids which could be formed in the same fashion by dividing the
  line <m>AC</m>.

\begin{center}
\begin{asy}
    size(300,30,keepAspect=false);
    
   draw((0.2,-0.03)--(0.2,0.03));
    label("<m>A</m>",(0.2,-0.03),S,fontsize(12pt));

    draw((0.752,-0.03)--(0.752,0.03));
    label("<m>B</m>",(0.752,-0.03),S,fontsize(12pt));
    
    draw((1.75,-0.03)--(1.75,0.03));
    label("<m>C</m>",(1.75,-0.03),S,fontsize(11pt));

    draw((0.2,0)--(1.75,0));
   
\end{asy}
\end{center}
Let us suppose, in algebraic notation, that <m>AC=b</m>, the unknown
<m>AB=a</m>; we will have <m>BC=b-a</m>, and the solid <m>a^2b-a^3</m> must satisfy
the proposed condition.

Now taking <m>a+e</m> in place of <m>a</m>, we have for the
solid <me>(a+e)^2(b-e-a)=ba^2+be^2+2bae-a^3-3ae^2-3a^2e-e^3.</me> I compare
this to the first solid: <m>a^2b-a^3</m>, as if they were equal, when in
fact they are not.


\begin{center} \ldots \end{center}

Then, I subtract the common terms from both sides, 

\begin{center} \ldots \end{center}

this done, one side of the equation has nothing, while the other
is <me>be^2+2bae-3ae^2-3a^2e-e^3.</me>

\begin{center} \ldots \end{center}

Dividing all terms by <m>e</m>, the <em>adequality</em> will hold between
<m>be+2ba</m> and <m>3ae+3a^2+e^2</m>.  After this division, if all terms may
again be divided by <m>e</m>, the division must be repeated, until there is
a term that can no longer be divided by <m>e</m>, or, to employ the
terminology of Vi\`{e}te\footnote{Fran\c{c}ois Vi\`{e}te (1540--1603)
  was a mathematician who worked as a codebreaker for several of the
  kings of France.  He introduced a system of symbolic algebra, which
  Fermat used and referenced here.  Vi\`{e}te used vowels for unknowns
  and consonants for knowns.  To our modern eyes, using <m>a</m> as an
  unknown instead of <m>x</m> might look a bit odd; this is because
  eventually Descartes' convention (using the letters <m>x,y,z</m> to
  represent unknowns) caught on rather than Vi\`{e}te's!}, a term
which is no longer affected by <m>e</m>.  But, in the proposed example, we
find that the division cannot be repeated; so, we have to stop there.

Now, I remove all the terms affected by <m>e</m>; on one side there remains
<m>2ba,</m> while the other has <m>3a^2,</m> terms between which it is necessary
to establish not a feigned comparison or an <em>adequality</em>, but
rather a true equation.  I divide both sides by <m>a</m>; giving me
<m>2b=3a</m>, or <m>b/a=3/2.</m>

Let us return to our original question, and divide <m>AC</m> at <m>B</m> such
that <m>AC/AB=3/2.</m> I say that the solid <m>AB^2\times BC</m> is the maximum
of all those which can be formed by dividing the line <m>AC</m>.
\end{source}

\begin{task}
\begin{enumerate}[(a)]
\item Check Fermat's work in the example above, filling in the details
  of the algebra that he glossed over.  Can you confirm each of his
  steps?

    \item Verify that Fermat's result matches what is produced by the
      modern method.  Specifically, maximize the
      function <me>f(x)=(b-x)x^2</me> on the interval <m>[0,b]</m>.

\item To see the equivalence of the two methods, let us once again
  compare with the limit definition of the derivative.  Take the
  function <m>f(x)=(b-x)x^2</m>, and instead of first calculating <m>f'(x)</m>
  and then setting that equal to zero, recall
  that
  <me>f'(x)=\lim_{\Delta x \to 0} \frac{f(x+\Delta x)-f(x)}{\Delta x},</me>
  so we should get the same result as if we had set
  <me>f(x+\Delta x)=f(x),</me> divided both sides by <m>\Delta x</m>, and then
  set all remaining <m>\Delta x</m> to zero.  Work this out to see if it
  matches what is produced by Fermat's method.
\end{enumerate}
\end{task}


\begin{task} Let us observe Fermat's results regarding ``{\sf all
    solids}'' by actually looking at a few solids!
\begin{enumerate}[(a)]
\item First, notice that when he said ``{\sf all solids}'', he was not
  talking about solids like balls, tetrahedra, etc.  What kinds of
  solids was he restricting his attention to?  How can you tell?
    \item Fermat claimed that to produce the biggest possible volume,
      one should ``{\sf divide <m>AC</m> at <m>B</m> such that <m>AC/AB=3/2.</m>}''
      Let us test this claim by working out some specific examples.
      In particular, choose the length <m>AC</m> to be equal to <m>12</m>.
      Then, try dividing the line <m>AC</m> four different ways, such that
      <m>AC/AB</m> has ratio <m>3/1</m>, <m>2/1</m>, <m>3/2</m>, and <m>1/1</m>.  Each time,
      draw a sketch of the resulting solid whose volume is
      <m>AB^2 \times BC</m>.  Label the edges and calculate the volumes.
      Which of those four solids has the biggest volume, and does that
      outcome agree with Fermat's claim?
\end{enumerate}
\end{task}


</subsubsection>
<subsubsection xml:id="Third">
<title>Third Example</title>


It appears that Fermat's method works fine, but both of the examples
we considered so far involved polynomial functions.  Maybe if we try a
function that is not a simple polynomial, then the method will fail!
Fermat wrote this example, titled <em>Appendix to the Method of
  Maxima and Minima</em>, in 1644 [\cite[page 136]{Tannery}]. 

\begin{source}
Radicals are often encountered in the course of working problems.

\begin{quote}
    \ldots
\it Given a semicircle of diameter <m>AB</m>, with perpendicular <m>DC</m> drawn
upon its diameter, find the maximum of the sum <m>AC+CD.</m> 
\end{quote}

\begin{center}
\vspace{.2in}

\begin{asy}
    size(300,150,keepAspect=true);

    import graph; 
    
    real f(real x) { return sqrt(1-(x-1)^2);}
    
    path g = graph(f,0,2);
    
    draw(g);
    
    label("<m>A</m>",(0,-0.03),S,fontsize(12pt));

    label("<m>B</m>",(2,-0.03),S,fontsize(12pt));
    
    label("<m>C</m>",(1.75,-0.03),S,fontsize(11pt));
    
    label("<m>D</m>",(1.75,sqrt(1-0.75^2)+0.2),S,fontsize(11pt));

    draw((1.75,0)--(1.75,sqrt(1-0.75^2)));

    draw((0,0)--(2,0));
   
\end{asy}

\end{center}

Let the diameter be taken as <m>b</m>, and let <m>AC=a</m>. We will thus have
<m>CD=\sqrt{ba-a^2}.</m> The question becomes the maximization of the
quantity <m>a+\sqrt{ba-a^2}</m>.



\end{source}

\begin{task}
  In trying to maximize the sum <m>AC+CD</m>, Fermat simply set <m>AC=a</m>.
  His formula for <m>CD</m>, however, takes some work to verify.
    \begin{enumerate}[(a)]
    \item Label the center of the circle as <m>E</m>.  Explain why the
      measure of <m>DE</m> is <m>b/2</m>.
        \item Explain why the measure of <m>CE</m> is <m>a-b/2.</m>
        \item Use the Pythagorean Theorem on <m>\triangle CDE</m> to
          calculate the length of <m>CD</m> in terms of <m>b</m> and <m>a</m>.
          Verify Fermat's formula for <m>CD</m>.
    \end{enumerate}
\end{task}

Pretty clearly, answering this question for a circle of a specific
size answers it for all circles, since the maximum length path would
scale with the radius of the circle.  Thus, for simplicity, we choose
to solve the problem in the case <m>b=1.</m>

\begin{task}
\begin{enumerate}[(a)]
\item Use Fermat's method to find the maximum of the quantity
  <m>a+\sqrt{a-a^2}.</m> That is, set up the adequality between
  <m>a+\sqrt{a-a^2}</m> and the same expression with <m>a+e</m> substituted for
  <m>a</m>.  Then continue to follow the steps in Fermat's method!

\item Use the modern method to confirm the answer that Fermat's method
  gives.  That is, use the chain rule to find the derivative of
  <m>f(x)=x+\sqrt{x-x^2}</m>. Then find the maximum by solving for the
  zeros of the derivative.  Also, identify the domain of <m>x</m>-values
  that are being considered.

\end{enumerate}
\end{task}

Let us call attention to one particularly nice aspect of Fermat's
method; it requires far less knowledge of derivatives than the modern
method.  For example, in the previous problem involving the expression
with the square root, we were able to eliminate the root by performing
the basic algebraic step of squaring both sides rather than needing to
evaluate the derivative of a square root function!



</subsubsection>
</subsection>
<subsection xml:id="no-label">
<title>Resolution</title>


The preceding examples have illustrated that Fermat's method is
actually very similar to the modern method, just written in different
notation.

\begin{task}\begin{enumerate}[(a)]
  \item Explain why Fermat's method and the modern method are
    essentially equivalent.  Where do they differ?
    \item Why does it make sense that Fermat's method would have had
      to rely more on algebra and less on analysis than the modern
      method? (For a hint, consider the year in which he was working!
      Do a bit of research and see if you can find who came up with
      our modern definitions of limits and derivatives, and when that
      happened!)
\end{enumerate}
 \end{task}

 Thus, Fermat's method was not laid out hastily, but rather was a
 lovely and valid mathematical method.  Descartes himself eventually
 agreed!  Descartes later said ``{\sf ...seeing the last method that
   you use for finding tangents to curved lines, I can reply to it in
   no other way than to say that it is very good and that, if you had
   explained it in this manner at the outset, I would have not
   contradicted it at all.}'' [\cite[page 192]{FermatBio}]

 \printbibliography[heading=subbibliography]


</subsection>
<subsection xml:id="no-label">
<title>Recommendations for Further Reading</title>


A fun and enriching comparison with Descartes' <em>method of
  normals</em> for optimization would be a great follow-up to this
project.  (Some suspect that Descartes' initial distaste for Fermat's
method was because it aimed to solve the same problem as his method of
normals, and was created at about the same time [\cite[pages 472 and
473]{Katz}].)  However, proper attention to Descartes' method is
likely to move well beyond the standard topics of a first-semester
calculus classroom.  To do this in detail might be better suited for a
multivariable calculus class, where one can appropriately discuss the
ideas of the normal vector and the radius of curvature.  To this end,
the author recommends Jerry Lodder's PSP <em>The Radius of Curvature
  According to Christiaan Hyugens</em> (available at {\tt
  https://digitalcommons.ursinus.edu/triumphs\_calculus/4}) in
addition to the description of Descartes' method given in <em>A
  History of Mathematics: An Introduction</em> [\cite{Katz}] cited above.

For students that are pursuing a degree in mathematics, this topic is
a perfect warmup to the eventual study of Abraham Robinson's theory of
nonstandard analysis (laid out beautifully in his 1966 work
<em>Non-standard Analysis</em> from Princeton University Press (1996),
ISBN 978-0-691-04490-3).  It could be worth mentioning that it is
possible to more formally prove the correctness of Fermat's method
using the hyperreal numbers, where Fermat's <m>e</m> represents an
infinitesimal.  However, to formally construct the aforementioned
number system requires a substantial amount of set theory and logic,
and it is probably an appropriate journey for junior or senior level
undergraduate studies at the earliest.



</subsection>
<subsection xml:id="no-label">
<title>Acknowledgments</title>

The author would like to thank the TRIUMPHS PIs and
 Advisory Board for the very helpful feedback throughout the writing
 of this project, in particular David Pengelley for bringing the
 author's attention to the bonus question suggested under Suggestions
 for Classroom Implementation.

 \end{refsection}










</subsection>
</section>
