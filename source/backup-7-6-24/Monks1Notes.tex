\section{Fermat's Method for Finding Maxima and Minima}
\markright{Fermat's Method for Finding Maxima and Minima: Notes to Instructors}

\subsection*{PSP Content: Topics and Goals}

This Primary Source Project (PSP) is intended to enrich an
introductory Calculus student's grasp on the definition of the
derivative and how it relates to finding maxima and minima of
functions.  The key competencies that come up in this project are as
follows:
\begin{itemize}
    \item Definition of the derivative
    \item Rules for calculating derivatives
    \item Tangents
    \item Optimization
\end{itemize}


\subsection*{Student Prerequisites}

In this project, we assume the student has already been exposed to the
limit definition of the derivative as well as the usual rules for
calculating derivatives (in particular, chain rule and power rule).
We also assume the student has been exposed to the Extreme Value
Theorem.

\subsection*{PSP Design, and Task Commentary}

This PSP will expose the student to the original, more algebraic
framework for finding extrema of functions.  Hopefully, seeing some of
the standard textbook exercises on maxima and minima (like Fermat's
example in Section \ref{First}) approached with a different method and
with different notation will break students out of recipe-thinking
with regards to optimization.

Fermat provided many more examples of his method of adequality
throughout his life, but many of these use rather sophisticated
constructions from geometry.  While beautiful in and of themselves,
the author feared that these would be too much of a departure from the
standard calculus curriculum.  The three examples chosen for this PSP
were purposefully selected because of their similarity to the types of
textbook optimization problems that are typically assigned in a
first-semester calculus course.

Section \ref{Third} is the one section where Fermat's original
solution via adequality is intentionally not shown.  The hope is that
by that point, the student can not only confirm Fermat's results using
the modern method, but can carry out Fermat's method as well!

Note that the final task does not ask for a rigorous proof of the
equivalence of the modern method to Fermat's method, but rather an
intuitive justification.  See the section below on Recommendations for
Further Reading for what such a proof would entail!

\subsection*{Suggestions for Classroom Implementation}

The author strongly suggests the instructor work through the entire
project before using it in class. In particular, it is easy to make a
simple error in the mess of Section \ref{Third}.


The reading and tasks of Section \ref{Adequality} make an ideal class
preparation assignment, while completion of the remainder of the PSP
might be more well-suited for a mix of in-class work and homework.


If the instructor desires an interesting wrap-up discussion for this
project, a peculiar phrase in the first primary source passage
provokes an interesting question.  While laying out his method, Fermat
said ``{\sf We will divide all the terms by $e$, {\bf or by a higher
    power of $e$}, such that on at least one of the sides, $e$ will
  disappear entirely.}''  This prompts a question: can it ever happen
that we divide by a higher power of $e$ rather than just $e$ itself?
Fermat's three examples included here only required division by a
single power of $e$, and even after further reading of Fermat's work,
the author was unable to find an example where Fermat divided by any
higher power of $e$.

The absence of such an example in Fermat's work is perhaps with good
reason!  Under mild assumptions (like the function in question having
a convergent power series on an interval containing the max/min one
seeks) one can show that only a constant function $f(a)$ could result
in the quantity $f(a+e)-f(a)$ being divisible by $e^2$.  For if it
were possible to write $f(a+e)-f(a)= e^2\cdot g(a,e)$ for some
polynomial $g(a,e)$ (possibly of infinite degree), then dividing both
sides by $e$ would produce $$\frac{f(a+e)-f(a)}{e}=e\cdot g(a,e).$$
Taking the limit of both sides as $e$ approaches zero
implies $$f'(a)=0$$ since $\lim_{e\to 0}g(a,e)$ converges to $g(a,0)$.
Since the derivative of $f$ is identically zero, $f$ must be a
constant function.

That analysis raises a further interesting question: why did Fermat
include that phrase regarding dividing by a higher power of $e$?  Was
Fermat simply unsure that it couldn't happen, and mentioned it in
passing just in case it ever did?  This seems plausible. Though our
proof above is not particularly difficult, it uses a heavy tool from a
toolbox that was unavailable to Fermat, namely the idea of a power
series expansion of a function.

Though it is unlikely we will be able to definitively resolve the
question of what Fermat's intents were with that phrase, having a
discussion like the one above could be a nice way to wrap up the
project with a class.  If nothing else, it can show the students the
fascinating thought exercises prompted by looking at primary sources!
It is hard to imagine such a question coming up in the context of
reading a polished modern textbook.

Copies of these PSPs are available at the TRIUMPHS website (see URL in
Acknowledgements).  The author is happy to provide \LaTeX\ code for
this project.  It was created using Overleaf which makes it convenient
to copy and share projects and can allow instructors to adapt this
project in whole or in part as they like for their course.


\subsection*{Sample Implementation Schedule (based on a 50 minute class period)}

This miniPSP can easily be implemented in one class period.  The
author has used this in the following manner, with good results:

\begin{itemize}
\item Assign students to read and complete tasks through the end of
  Section \ref{Adequality} as a class preparation assignment.
\item Begin class with 10 minutes to have students share a few of the
  observations they came up with when comparing/contrasting the
  methods. and hold a discussion based on those questions, ideally
  with the primary source on the projector in front of you.
\item Allow them to work through the PSP for the next 35 minutes in
  small groups as you and/or learning assistants walk through the
  classroom and help.
\item In the last 5 minutes, it is sometimes nice to call the students
  together to regroup for a brief discussion.  See if anyone has
  thoughts on why Fermat's method and the modern method are
  equivalent!  It may be helpful to call their attention to the idea
  that $f(x+\Delta x)$ and $f(x)$ are very close to being equal for
  very small $\Delta x$ if $f(x)$ has a maximum or minimum at $x$ (and
  perhaps draw a picture to this effect on the board).  It can also be
  a nice followup to mention that the method does not seem to have a
  way to distinguish saddle points!
\item The students can complete all remaining unfinished tasks for
  homework.  Note that it is likely they will still have most of
  Sections \ref{Second} and \ref{Third} to complete, but this should
  be doable for homework if they successfully made it through Section
  \ref{First}.
\end{itemize}

 

